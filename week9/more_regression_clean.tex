% Options for packages loaded elsewhere
% Options for packages loaded elsewhere
\PassOptionsToPackage{unicode}{hyperref}
\PassOptionsToPackage{hyphens}{url}
\PassOptionsToPackage{dvipsnames,svgnames,x11names}{xcolor}
%
\documentclass[
  letterpaper,
  DIV=11,
  numbers=noendperiod]{scrartcl}
\usepackage{xcolor}
\usepackage{amsmath,amssymb}
\setcounter{secnumdepth}{-\maxdimen} % remove section numbering
\usepackage{iftex}
\ifPDFTeX
  \usepackage[T1]{fontenc}
  \usepackage[utf8]{inputenc}
  \usepackage{textcomp} % provide euro and other symbols
\else % if luatex or xetex
  \usepackage{unicode-math} % this also loads fontspec
  \defaultfontfeatures{Scale=MatchLowercase}
  \defaultfontfeatures[\rmfamily]{Ligatures=TeX,Scale=1}
\fi
\usepackage{lmodern}
\ifPDFTeX\else
  % xetex/luatex font selection
\fi
% Use upquote if available, for straight quotes in verbatim environments
\IfFileExists{upquote.sty}{\usepackage{upquote}}{}
\IfFileExists{microtype.sty}{% use microtype if available
  \usepackage[]{microtype}
  \UseMicrotypeSet[protrusion]{basicmath} % disable protrusion for tt fonts
}{}
\makeatletter
\@ifundefined{KOMAClassName}{% if non-KOMA class
  \IfFileExists{parskip.sty}{%
    \usepackage{parskip}
  }{% else
    \setlength{\parindent}{0pt}
    \setlength{\parskip}{6pt plus 2pt minus 1pt}}
}{% if KOMA class
  \KOMAoptions{parskip=half}}
\makeatother
% Make \paragraph and \subparagraph free-standing
\makeatletter
\ifx\paragraph\undefined\else
  \let\oldparagraph\paragraph
  \renewcommand{\paragraph}{
    \@ifstar
      \xxxParagraphStar
      \xxxParagraphNoStar
  }
  \newcommand{\xxxParagraphStar}[1]{\oldparagraph*{#1}\mbox{}}
  \newcommand{\xxxParagraphNoStar}[1]{\oldparagraph{#1}\mbox{}}
\fi
\ifx\subparagraph\undefined\else
  \let\oldsubparagraph\subparagraph
  \renewcommand{\subparagraph}{
    \@ifstar
      \xxxSubParagraphStar
      \xxxSubParagraphNoStar
  }
  \newcommand{\xxxSubParagraphStar}[1]{\oldsubparagraph*{#1}\mbox{}}
  \newcommand{\xxxSubParagraphNoStar}[1]{\oldsubparagraph{#1}\mbox{}}
\fi
\makeatother

\usepackage{color}
\usepackage{fancyvrb}
\newcommand{\VerbBar}{|}
\newcommand{\VERB}{\Verb[commandchars=\\\{\}]}
\DefineVerbatimEnvironment{Highlighting}{Verbatim}{commandchars=\\\{\}}
% Add ',fontsize=\small' for more characters per line
\usepackage{framed}
\definecolor{shadecolor}{RGB}{241,243,245}
\newenvironment{Shaded}{\begin{snugshade}}{\end{snugshade}}
\newcommand{\AlertTok}[1]{\textcolor[rgb]{0.68,0.00,0.00}{#1}}
\newcommand{\AnnotationTok}[1]{\textcolor[rgb]{0.37,0.37,0.37}{#1}}
\newcommand{\AttributeTok}[1]{\textcolor[rgb]{0.40,0.45,0.13}{#1}}
\newcommand{\BaseNTok}[1]{\textcolor[rgb]{0.68,0.00,0.00}{#1}}
\newcommand{\BuiltInTok}[1]{\textcolor[rgb]{0.00,0.23,0.31}{#1}}
\newcommand{\CharTok}[1]{\textcolor[rgb]{0.13,0.47,0.30}{#1}}
\newcommand{\CommentTok}[1]{\textcolor[rgb]{0.37,0.37,0.37}{#1}}
\newcommand{\CommentVarTok}[1]{\textcolor[rgb]{0.37,0.37,0.37}{\textit{#1}}}
\newcommand{\ConstantTok}[1]{\textcolor[rgb]{0.56,0.35,0.01}{#1}}
\newcommand{\ControlFlowTok}[1]{\textcolor[rgb]{0.00,0.23,0.31}{\textbf{#1}}}
\newcommand{\DataTypeTok}[1]{\textcolor[rgb]{0.68,0.00,0.00}{#1}}
\newcommand{\DecValTok}[1]{\textcolor[rgb]{0.68,0.00,0.00}{#1}}
\newcommand{\DocumentationTok}[1]{\textcolor[rgb]{0.37,0.37,0.37}{\textit{#1}}}
\newcommand{\ErrorTok}[1]{\textcolor[rgb]{0.68,0.00,0.00}{#1}}
\newcommand{\ExtensionTok}[1]{\textcolor[rgb]{0.00,0.23,0.31}{#1}}
\newcommand{\FloatTok}[1]{\textcolor[rgb]{0.68,0.00,0.00}{#1}}
\newcommand{\FunctionTok}[1]{\textcolor[rgb]{0.28,0.35,0.67}{#1}}
\newcommand{\ImportTok}[1]{\textcolor[rgb]{0.00,0.46,0.62}{#1}}
\newcommand{\InformationTok}[1]{\textcolor[rgb]{0.37,0.37,0.37}{#1}}
\newcommand{\KeywordTok}[1]{\textcolor[rgb]{0.00,0.23,0.31}{\textbf{#1}}}
\newcommand{\NormalTok}[1]{\textcolor[rgb]{0.00,0.23,0.31}{#1}}
\newcommand{\OperatorTok}[1]{\textcolor[rgb]{0.37,0.37,0.37}{#1}}
\newcommand{\OtherTok}[1]{\textcolor[rgb]{0.00,0.23,0.31}{#1}}
\newcommand{\PreprocessorTok}[1]{\textcolor[rgb]{0.68,0.00,0.00}{#1}}
\newcommand{\RegionMarkerTok}[1]{\textcolor[rgb]{0.00,0.23,0.31}{#1}}
\newcommand{\SpecialCharTok}[1]{\textcolor[rgb]{0.37,0.37,0.37}{#1}}
\newcommand{\SpecialStringTok}[1]{\textcolor[rgb]{0.13,0.47,0.30}{#1}}
\newcommand{\StringTok}[1]{\textcolor[rgb]{0.13,0.47,0.30}{#1}}
\newcommand{\VariableTok}[1]{\textcolor[rgb]{0.07,0.07,0.07}{#1}}
\newcommand{\VerbatimStringTok}[1]{\textcolor[rgb]{0.13,0.47,0.30}{#1}}
\newcommand{\WarningTok}[1]{\textcolor[rgb]{0.37,0.37,0.37}{\textit{#1}}}

\usepackage{longtable,booktabs,array}
\usepackage{calc} % for calculating minipage widths
% Correct order of tables after \paragraph or \subparagraph
\usepackage{etoolbox}
\makeatletter
\patchcmd\longtable{\par}{\if@noskipsec\mbox{}\fi\par}{}{}
\makeatother
% Allow footnotes in longtable head/foot
\IfFileExists{footnotehyper.sty}{\usepackage{footnotehyper}}{\usepackage{footnote}}
\makesavenoteenv{longtable}
\usepackage{graphicx}
\makeatletter
\newsavebox\pandoc@box
\newcommand*\pandocbounded[1]{% scales image to fit in text height/width
  \sbox\pandoc@box{#1}%
  \Gscale@div\@tempa{\textheight}{\dimexpr\ht\pandoc@box+\dp\pandoc@box\relax}%
  \Gscale@div\@tempb{\linewidth}{\wd\pandoc@box}%
  \ifdim\@tempb\p@<\@tempa\p@\let\@tempa\@tempb\fi% select the smaller of both
  \ifdim\@tempa\p@<\p@\scalebox{\@tempa}{\usebox\pandoc@box}%
  \else\usebox{\pandoc@box}%
  \fi%
}
% Set default figure placement to htbp
\def\fps@figure{htbp}
\makeatother





\setlength{\emergencystretch}{3em} % prevent overfull lines

\providecommand{\tightlist}{%
  \setlength{\itemsep}{0pt}\setlength{\parskip}{0pt}}



 


\KOMAoption{captions}{tableheading}
\makeatletter
\@ifpackageloaded{caption}{}{\usepackage{caption}}
\AtBeginDocument{%
\ifdefined\contentsname
  \renewcommand*\contentsname{Table of contents}
\else
  \newcommand\contentsname{Table of contents}
\fi
\ifdefined\listfigurename
  \renewcommand*\listfigurename{List of Figures}
\else
  \newcommand\listfigurename{List of Figures}
\fi
\ifdefined\listtablename
  \renewcommand*\listtablename{List of Tables}
\else
  \newcommand\listtablename{List of Tables}
\fi
\ifdefined\figurename
  \renewcommand*\figurename{Figure}
\else
  \newcommand\figurename{Figure}
\fi
\ifdefined\tablename
  \renewcommand*\tablename{Table}
\else
  \newcommand\tablename{Table}
\fi
}
\@ifpackageloaded{float}{}{\usepackage{float}}
\floatstyle{ruled}
\@ifundefined{c@chapter}{\newfloat{codelisting}{h}{lop}}{\newfloat{codelisting}{h}{lop}[chapter]}
\floatname{codelisting}{Listing}
\newcommand*\listoflistings{\listof{codelisting}{List of Listings}}
\makeatother
\makeatletter
\makeatother
\makeatletter
\@ifpackageloaded{caption}{}{\usepackage{caption}}
\@ifpackageloaded{subcaption}{}{\usepackage{subcaption}}
\makeatother
\usepackage{bookmark}
\IfFileExists{xurl.sty}{\usepackage{xurl}}{} % add URL line breaks if available
\urlstyle{same}
\hypersetup{
  pdftitle={Section materials: More Regression with Maternal Smoking and Infant Birthweight},
  pdfauthor={scott cunningham},
  colorlinks=true,
  linkcolor={blue},
  filecolor={Maroon},
  citecolor={Blue},
  urlcolor={Blue},
  pdfcreator={LaTeX via pandoc}}


\title{Section materials: More Regression with Maternal Smoking and
Infant Birthweight}
\author{scott cunningham}
\date{}
\begin{document}
\maketitle


\subsection{Background}\label{background}

This week we deepen our understanding of the mechanics of Ordinary Least
Squares (OLS). We'll continue using the \textbf{Cattaneo (2010)} dataset
that examines the effect of maternal smoking (\texttt{mbsmoke}) on
infant birthweight (\texttt{bweight}), but we will also look at other
variables so that students have practice understanding the mechanics of
OLS in a context that they are by now familiar with. Our goal is to
\textbf{see how OLS solves for coefficients} without matrix
algebra---first by hand with simple formulas, then by confirming that
lm() produces identical results.

\subsection{Reviewing with bivariate
regression}\label{reviewing-with-bivariate-regression}

\subsection{Question 1 (4 points)}\label{question-1-4-points}

First, we will begin by importing the dataset from Cunningham's github
repo with the \texttt{read\_dta} command from the \texttt{haven}
package. Then we will regress \texttt{bweight} onto \texttt{mbsmoke}.
They did this last week, but I think they need to keep doing it, keep
practicing the interpretation.

\begin{Shaded}
\begin{Highlighting}[]
\NormalTok{data }\OtherTok{\textless{}{-}} \FunctionTok{read\_dta}\NormalTok{(}\StringTok{"https://raw.github.com/scunning1975/mixtape/master/cattaneo2.dta"}\NormalTok{)}
\end{Highlighting}
\end{Shaded}

Notice that we immediately estimated the effect of maternal smoking on
birthweight. Students are encouraged at this point to correctly
interpret the coefficient. I would like for you to help them understand
that the regression coefficient when the treatment is a dummy is
interpreted as \textbf{percentage point differences}. It's important
that they learn to use the correct units -- that these are
\textbf{percentage points}, not \textbf{percentage changes}.

\begin{enumerate}
\def\labelenumi{\arabic{enumi}.}
\item
  Interpret the slope and the intercept.
\item
  Use the intercept and the slope coefficient to calculate the treatment
  group outcome mean.
\end{enumerate}

\subsection{Illustration of covariance and
variance.}\label{illustration-of-covariance-and-variance.}

Next we will use R as a calculator as we help students understand
\emph{how} OLS calculates the slope and intercept coefficients. Remember
in a simple regression model:

\(Y_i = \beta_0 + \beta_1 X_i + \varepsilon_i\)

We have to two unknown coefficients -- the \(\beta_0\) and \(\beta_1\).
Recall that we call \(\beta_0\) the intercept which is the mean outcome
for the comparison group (i.e., when \(X=0\)). And we call
\(\varepsilon\) the error term which is all other causes of the outcome
that is not \(X\).

We discussed in class that the slope coefficient formula that OLS uses
is a ``scaled covariance'':

\(\widehat{\beta_1} = \frac{Cov(X,Y)}{Var(X)}\)

But the intercept also has a formula and it is:

\(\widehat{\beta_0} = \bar{Y} - \widehat{\beta_1}\bar{X}\)

So our goal here is to \emph{manually} using R as a calculator calculate
those two terms. So that you have these formula here with you, the
equations for calculating covariance and variance are:

\(Cov(X,Y) = E[XY] - E[X]E[Y]\)

\(Var(X) = E[X^2] - E[X]^2\)

We will do this in steps. First, let's calculate the covariance and
variance terms. First, we will just rename \texttt{bweight} and
\texttt{mbsmoke} as \texttt{Y} and \texttt{X} so that you can trace it
back to those formula more easily.

\begin{Shaded}
\begin{Highlighting}[]
\NormalTok{Y }\OtherTok{\textless{}{-}} 
\NormalTok{X }\OtherTok{\textless{}{-}} 
\end{Highlighting}
\end{Shaded}

\begin{verbatim}
Error in parse(text = input): <text>:3:0: unexpected end of input
1: Y <- 
2: X <- 
  ^
\end{verbatim}

Next we will calculate the expectations.

\begin{Shaded}
\begin{Highlighting}[]
\NormalTok{EX   }\OtherTok{\textless{}{-}}             \CommentTok{\# E[X]}
\NormalTok{EY   }\OtherTok{\textless{}{-}}             \CommentTok{\# E[Y]}
\NormalTok{EXY  }\OtherTok{\textless{}{-}}         \CommentTok{\# E[XY]}
\NormalTok{EX2  }\OtherTok{\textless{}{-}}         \CommentTok{\# E[X\^{}2]}
\end{Highlighting}
\end{Shaded}

\begin{verbatim}
Error in parse(text = input): <text>:5:0: unexpected end of input
3: EXY  <-         # E[XY]
4: EX2  <-         # E[X^2]
  ^
\end{verbatim}

Then we will build those covariances and variances.

\begin{Shaded}
\begin{Highlighting}[]
\NormalTok{Cov\_XY }\OtherTok{\textless{}{-}} 
\NormalTok{Var\_X  }\OtherTok{\textless{}{-}} 
\end{Highlighting}
\end{Shaded}

\begin{verbatim}
Error in parse(text = input): <text>:3:0: unexpected end of input
1: Cov_XY <- 
2: Var_X  <- 
  ^
\end{verbatim}

And finally, we calculate those OLS coefficients. We will now rebuild
these quantities using R as a calculator (R does a lot of things) so you
can see exactly how \texttt{lm()} arrived at those numbers.

\begin{Shaded}
\begin{Highlighting}[]
\NormalTok{b1\_hat }\OtherTok{\textless{}{-}} 
\NormalTok{b0\_hat }\OtherTok{\textless{}{-}} 
\FunctionTok{c}\NormalTok{(}\AttributeTok{b0\_hat =}\NormalTok{ b0\_hat, }\AttributeTok{b1\_hat =}\NormalTok{ b1\_hat)}
\end{Highlighting}
\end{Shaded}

\begin{verbatim}
Error: object 'b0_hat' not found
\end{verbatim}

Now let's compare this with what we did earlier with the \texttt{lm()}
command.

\begin{Shaded}
\begin{Highlighting}[]
          \CommentTok{\# OLS results from lm()}
        \CommentTok{\# our manual slope (Cov/Var)}
       \CommentTok{\# our manual intercept}
\end{Highlighting}
\end{Shaded}

\subsection{Question 2 (10 points)}\label{question-2-10-points}

Now it's your turn. Redo what we did, but use instead the following
regressions.

\begin{enumerate}
\def\labelenumi{\arabic{enumi}.}
\tightlist
\item
  Regress \texttt{bweight} onto \texttt{mage}, calculate the
  coefficients manually, and compare it with the \texttt{lm()} output.
\end{enumerate}

\begin{Shaded}
\begin{Highlighting}[]
\NormalTok{fit2 }\OtherTok{\textless{}{-}} 
 \CommentTok{\# regression coefficients}

\CommentTok{\# manual calculation}
\NormalTok{Y }\OtherTok{\textless{}{-}} 
\NormalTok{X }\OtherTok{\textless{}{-}} 

\NormalTok{EX   }\OtherTok{\textless{}{-}}             \CommentTok{\# E[X]}
\NormalTok{EY   }\OtherTok{\textless{}{-}}             \CommentTok{\# E[Y]}
\NormalTok{EXY  }\OtherTok{\textless{}{-}}         \CommentTok{\# E[XY]}
\NormalTok{EX2  }\OtherTok{\textless{}{-}}         \CommentTok{\# E[X\^{}2]}

\NormalTok{Cov\_XY }\OtherTok{\textless{}{-}} 
\NormalTok{Var\_X  }\OtherTok{\textless{}{-}} 

\NormalTok{b1\_hat }\OtherTok{\textless{}{-}} 
\NormalTok{b0\_hat }\OtherTok{\textless{}{-}} 
\FunctionTok{c}\NormalTok{(}\AttributeTok{b0\_hat =}\NormalTok{ b0\_hat, }\AttributeTok{b1\_hat =}\NormalTok{ b1\_hat)}
\end{Highlighting}
\end{Shaded}

\begin{verbatim}
Error: object 'b0_hat' not found
\end{verbatim}

\begin{Shaded}
\begin{Highlighting}[]
          \CommentTok{\# OLS results from lm()}
        \CommentTok{\# our manual slope (Cov/Var)}
        \CommentTok{\# our manual intercept}
\end{Highlighting}
\end{Shaded}

\begin{enumerate}
\def\labelenumi{\arabic{enumi}.}
\setcounter{enumi}{1}
\tightlist
\item
  Regress \texttt{bweight} onto \texttt{fedu}, calculate the
  coefficients manually, and compare it with the \texttt{lm()} output.
\end{enumerate}

\begin{Shaded}
\begin{Highlighting}[]
\NormalTok{fit3 }\OtherTok{\textless{}{-}} 
 \CommentTok{\# regression coefficients}

\CommentTok{\# manual calculation}
\NormalTok{Y }\OtherTok{\textless{}{-}} 
\NormalTok{X }\OtherTok{\textless{}{-}} 

\NormalTok{EX   }\OtherTok{\textless{}{-}}            \CommentTok{\# E[X]}
\NormalTok{EY   }\OtherTok{\textless{}{-}}             \CommentTok{\# E[Y]}
\NormalTok{EXY  }\OtherTok{\textless{}{-}}         \CommentTok{\# E[XY]}
\NormalTok{EX2  }\OtherTok{\textless{}{-}}         \CommentTok{\# E[X\^{}2]}

\NormalTok{Cov\_XY }\OtherTok{\textless{}{-}} 
\NormalTok{Var\_X  }\OtherTok{\textless{}{-}} 

\NormalTok{b1\_hat }\OtherTok{\textless{}{-}} 
\NormalTok{b0\_hat }\OtherTok{\textless{}{-}} 
\FunctionTok{c}\NormalTok{(}\AttributeTok{b0\_hat =}\NormalTok{ b0\_hat, }\AttributeTok{b1\_hat =}\NormalTok{ b1\_hat)}
\end{Highlighting}
\end{Shaded}

\begin{verbatim}
Error: object 'b0_hat' not found
\end{verbatim}

\begin{Shaded}
\begin{Highlighting}[]
          \CommentTok{\# OLS results from lm()}
        \CommentTok{\# our manual slope (Cov/Var)}
        \CommentTok{\# our manual intercept}
\end{Highlighting}
\end{Shaded}

\begin{enumerate}
\def\labelenumi{\arabic{enumi}.}
\setcounter{enumi}{2}
\tightlist
\item
  If OLS is just a scaled covariance (scaled by the variance), then what
  does the sign and size of \texttt{Cov(X,Y)} tell us about the
  relationship between smoking and birthweight?
\end{enumerate}

\subsection{\texorpdfstring{Question 3 (8 points): Visualizing Variance
and Mean with
\texttt{ggplot}}{Question 3 (8 points): Visualizing Variance and Mean with ggplot}}\label{question-3-8-points-visualizing-variance-and-mean-with-ggplot}

Next, let's visualize this. Variance measures how far data points are
spread around their mean, and since both variance and mean were in our
regression coefficients, this is a good time to introduce them both
visually. Let's use \texttt{mage} (maternal age) to visualize what this
means since you just calculated a regression equation with it as your
right-hand-side variable.

\begin{Shaded}
\begin{Highlighting}[]
\FunctionTok{ggplot}\NormalTok{(data, }\FunctionTok{aes}\NormalTok{(}\AttributeTok{x =}\NormalTok{ mage)) }\SpecialCharTok{+}
  \FunctionTok{geom\_histogram}\NormalTok{(}\AttributeTok{binwidth =} \DecValTok{1}\NormalTok{, }\AttributeTok{fill =} \StringTok{"skyblue"}\NormalTok{, }\AttributeTok{color =} \StringTok{"white"}\NormalTok{, }\AttributeTok{boundary =} \DecValTok{0}\NormalTok{) }\SpecialCharTok{+}
  \FunctionTok{geom\_vline}\NormalTok{(}\FunctionTok{aes}\NormalTok{(}\AttributeTok{xintercept =} \FunctionTok{mean}\NormalTok{(mage, }\AttributeTok{na.rm =} \ConstantTok{TRUE}\NormalTok{)),}
             \AttributeTok{color =} \StringTok{"red"}\NormalTok{, }\AttributeTok{linetype =} \StringTok{"dashed"}\NormalTok{, }\AttributeTok{linewidth =} \DecValTok{1}\NormalTok{) }\SpecialCharTok{+}
  \FunctionTok{labs}\NormalTok{(}
    \AttributeTok{title =} \StringTok{"Distribution of Maternal Age (mage)"}\NormalTok{,}
    \AttributeTok{subtitle =} \StringTok{"The red dashed line shows the mean.  Variance reflects how spread out the ages are around this mean."}\NormalTok{,}
    \AttributeTok{x =} \StringTok{"Maternal age"}\NormalTok{,}
    \AttributeTok{y =} \StringTok{"Count"}
\NormalTok{  ) }\SpecialCharTok{+}
  \FunctionTok{theme\_minimal}\NormalTok{(}\AttributeTok{base\_size =} \DecValTok{13}\NormalTok{)}
\end{Highlighting}
\end{Shaded}

\pandocbounded{\includegraphics[keepaspectratio]{more_regression_clean_files/figure-pdf/unnamed-chunk-9-1.pdf}}

\begin{enumerate}
\def\labelenumi{\arabic{enumi}.}
\item
  What does the height and width of each bar represent?
\item
  Where is most of the data concentrated?
\item
  If the bars were more tightly bunched around the red line, how would
  that change the variance?
\item
  How might ``high'' or ``low'' variance in \texttt{mage} affect our
  regression coefficient estimates?
\end{enumerate}

\subsection{\texorpdfstring{Question 4 (9 points): Visualizing Variance
and Standard Deviation with
\texttt{ggplot}}{Question 4 (9 points): Visualizing Variance and Standard Deviation with ggplot}}\label{question-4-9-points-visualizing-variance-and-standard-deviation-with-ggplot}

Now that students have seen the regression coefficient as a covariance
scaled by the variance, we will introduce them to the concept of
standard deviation via visualization.

The variance is the \emph{average squared distance from the mean}. And
the standard deviation is just its square root, which puts it back in
the same units as the mean itself. Our hope is that they will be able to
understand conceptually variance, mean and standard deviation better if
they can see it visualized, and since we just covered the regression
coefficient, our hope is that this is laddering off the previous idea.

We write down the standard deviation as follows:

\(s_X = \sqrt{Var(X)} = \sqrt{\frac{\sum (X_i-\bar X)^2}{n-1}}\)

And then we recreate the previous figure, but now we lay on top of the
variance graph a vertical line for the mean, and two lines for the
standard deviation.

\begin{Shaded}
\begin{Highlighting}[]
\NormalTok{sd\_mage }\OtherTok{\textless{}{-}} 
\NormalTok{mean\_mage }\OtherTok{\textless{}{-}} 
\end{Highlighting}
\end{Shaded}

\begin{verbatim}
Error in parse(text = input): <text>:3:0: unexpected end of input
1: sd_mage <- 
2: mean_mage <- 
  ^
\end{verbatim}

\begin{enumerate}
\def\labelenumi{\arabic{enumi}.}
\tightlist
\item
  How much of the data is within one standard deviation of the mean?
\end{enumerate}

\begin{Shaded}
\begin{Highlighting}[]
\CommentTok{\# Using mage}
\NormalTok{X }\OtherTok{\textless{}{-}} 

\NormalTok{mean\_X }\OtherTok{\textless{}{-}} 
\NormalTok{sd\_X   }\OtherTok{\textless{}{-}} 

\NormalTok{lower }\OtherTok{\textless{}{-}} 
\NormalTok{upper }\OtherTok{\textless{}{-}} 

\CommentTok{\# Calculate proportion of observations within 1 SD of mean}
\NormalTok{within\_1sd }\OtherTok{\textless{}{-}} 
\NormalTok{within\_1sd}
\end{Highlighting}
\end{Shaded}

\begin{verbatim}
Error: object 'within_1sd' not found
\end{verbatim}

\begin{enumerate}
\def\labelenumi{\arabic{enumi}.}
\setcounter{enumi}{1}
\tightlist
\item
  Why is the standard deviation easier to interpret than the variance?
\end{enumerate}

\begin{Shaded}
\begin{Highlighting}[]
\NormalTok{sd\_mage }\OtherTok{\textless{}{-}} 
\NormalTok{var\_mage }\OtherTok{\textless{}{-}} 
\FunctionTok{cat}\NormalTok{(}\StringTok{"Variance:"}\NormalTok{, var\_mage, }\StringTok{"}\SpecialCharTok{\textbackslash{}n}\StringTok{Standard deviation:"}\NormalTok{, sd\_mage)}
\end{Highlighting}
\end{Shaded}

\begin{verbatim}
Error: object 'var_mage' not found
\end{verbatim}

\subsection{Conclusion}\label{conclusion}

That concludes the basics of regression mechanics. In the next Section,
we will review two more things: 1) multivariate regression
\emph{calculations} without matrix algebra and 2) \emph{sampling}. We
will also start working towards hypothesis testing. But for now, this
week's Section has been focused on helping students calculate regression
coefficients by hand, using R as a calculator, visualizing mean,
variance and standard deviations.




\end{document}
